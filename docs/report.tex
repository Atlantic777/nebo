\documentclass[a4paper]{article}
\usepackage[a4paper,margin=4cm]{geometry}
\usepackage[utf8]{inputenc}
\usepackage[T1]{fontenc}
\usepackage{tikz}
\usepackage{listings}
\usepackage[serbian]{babel}

\lstset{
  frame=single,
  numbers=left,
}

\begin{document}

\section{Opis zadatka}

Konstruisati distribuirani sistem za računanje u oblaku. Sistem se sastoji od
udaljenog poslužioca (jednog ili više) i korisničkog programa za zadavanje
zahteva za obradu. Korisnički program i poslužilac razmenjuju podatke pomoću
deljene memorije, a poruke pomoću imenovanih redova za razmenu poruka. Kao
primer upotrebe konstruisati sistem koji omogućava prebrojavanje zadatih reči u
proizvoljnoj tekstualnoj datoteci.

Ovaj sistem treba da se izvršava u Amazon oblaku. Za deljenu memoriju
iskoristiti S3 uslugu, a za razmenu poruka SQS. Poslužilac treba da se izvršava
u virtuelnoj mašini, deo EC2 usluge. Potrebno je automatizovati pokretanje i
upotrebu sistema pomoću {\em boto3} biblioteke za upravljanje Amazon oblakom iz
programskog jezika \mbox{\em Python}.

\section{Predloženo rešenje}

Korisniku je omogućeno da kreira novi sistem i zada proceduru za obradu.
Omogućena je paralelna obrada, ali bez saradnje odnosno jedan zahtev obrađuje
samo jedna jedinica za obradu. Paralelizam se javlja na nivou jedne instance
(više procesorskih jezgara) i na nivou više EC2 instanci (više virtuelnih
mašina).

Pri pokretanju servisa klijent postavlja datoteku sa procedurom za obradu u
deljenu memoriju. Po pokretanju servisa ta datoteka je preuzeta iz deljene
memorije i postavljena u lokalnu memoriju virtuelne mašine. Postoji jedan red za
razmenu poruka koji služi za izdavanje novih zahteva za obradu. Iz tog reda više
virtuelnih mašina može istovremeno da preuzima i obrađuje različite zahteve.
Rezultat obrade se smešta u izlaznu datoteku i postavlja u deljenu memoriju.
Putanja do datoteke sa rezultatom je 

\section{Izvedba rešenja}

Sistem je podeljen u tri sloja:

\begin{enumerate}
  \item nebo.aws - Komponente nižeg nivoa za upravljanje resursima u oblaku.
  \item nebo.core - Jezgro, funkcionalne jedinice visokog nivoa.
  \item nebo.cli - Konzolna korisnička sprega.
\end{enumerate}

\subsection{nebo.aws}

Ova komponenta sistema sadrži rukovaoce resursima. Oslanja se neposredno na
boto3 biblioteku. Omogućeno je upravlanje sledećim resursima:

\begin{itemize}
\item EC2 - virtuelne mašine
\item S3 - deljena memorija
\item SQS - komunikacioni kanali
\end{itemize}

Funkcionalnisti su dodavane po potrebi i generalizacija nije bila u cilju. Ovo
je pomoćna biblioteka namenjena samo za upotrebu u ovom projektu. Sledi pregled
funkcionalnosti za sva tri rukovaoca.

{\bf EC2Handler} je rukovalac koji omogućava kreiranje novih virtuelnih mašina
ili upravljanje postojećim. Za upravljanje postojećom instancom potrebno je
proslediti konstruktoru parametar InstanceId. Pri kreiranju novih virtuelnih
mašina moguće je izabrati tip insetance prema raspoloživim resursima (količina
radne memorije, broj procesora i drugo) i otisak operativnog sistema (system
image, AMI). Podrazumevani tip instance je t2.micro, a podrazumevani otisak
operativnog sistema je Amazon Linux (ami-bf2ba8d0) koji su određeni u
nebo/aws/ec2.py datoteci. Kreiranje EC2Handler objekta i kreiranje instance su
dva radzvojena koraka. Zahtev za kreiranje virtulne mašine je izdat tek nakon
poziva new\_instance() metode koji je blokirajuć i završiće se tek nakon što je
mašina potpuno funkcionalna (instance status ok). Pre pokretanja nove mašine
moguće je postaviti korisničku ``skriptu'' (user data) koja je izvršena po
pokretanju sistema. Omogućeno je upravljanje postojećim virtuelnim mašinama
(zaustavljanje, pokretanje, uklanjanje, opširan opis, provera da li još uvek
postoji).

{\bf S3Handler} je rukovalac namenjen za upravljanje S3 uslugama, odnosno
deljenom memorijom. Deljena memorija u ovom sistemu je određena nazivom servisa
i lokacijom odnosno prefiksom. Prefiks ima ulogu logičkog poddirektorijuma
unutar S3 korpe. Omogućene su operacije psotavljanje datoteke na zadatu
lokaciju, dobavljanje datoteke, pražnjenje i uklanjanje korpe, čekanje da se
objekat pojavi na očekivanoj lokaciji, dobavljanje (potpisane) adrese za
preuzimanje datoteke (presigned URL) i pospremanje korpe odnosno lokacije
(brisanje zaostalih datoteka, garbage collection). Napomena: nije neophodno
izričito napraviti novu korpu već je to podrazumevano i izvodi se samo ukoliko
je potrebno.

{\bf SQSHandler} je rukovalac namanjen za upravljanje SQS uslugama, redovima za
razmenu poruka. Slično kao i za S3Handler, moguće je napraviti red koji se
odnosi na servis i dodatno ima naziv. Omogućeno je slanje i prijem tačno jedne
poruke po pozivu odgovarajuće metode.

\subsection{nebo.core}

{\bf NeboService} objedinjuje korake za stvaranje novog servisa. Ovi koraci
podrazumevaju sledeće operacije:

\begin{enumerate}
\item Kreiranje odgovarajućih korpi u deljenoj memoriji.
\item Postavljanje korisničkog programa u deljenu memoriju.
\item Priprema skripte za inicijalizaciju (init\_setup.sh)
\item Pokretanje nove EC2 instance.
\end{enumerate}

{\bf NeboClient} objedinjuje korake za izdavanje novog zahteva za obradu. Ovi
koraci su:

\begin{enumerate}
\item Postavljanje ulazne datoteke u deljenu memoriju servisa.
\item Izdavanje novog zahteva za obradu putem reda za razmenu poruka.
\item Čekanje odgovora na povratnom kanalu.
\end{enumerate}

{\bf NeboServer} je samostalna komponenta za obradu novih korisničkih zahteva i
uklanjane zaostalih datoteka iz deljne memorije. Ova komponenta se izvršava na
svakoj EC2 instanci i sastoji iz dva dela, poslužioca i korisničkog programa.
Poslužilac je zajednički za sve EC2 instance, nezavisno od korisničkog programa
i vrste servisa. NeboServer je komponenta koja može da pokrene više procesa za
obradu (Worker) u kojima će biti obrađeni pristigli zahtevi pomoću korisničkog
programa. Korisnički program se pri pokretanju instance dobavlja iz deljene
memorije i učitava pri inicijalizaciji NeboServer komponente.

Obrada jednog zahteva sastoji se od sledećih koraka:

\begin{enumerate}
\item Čitanje jednog zahteva i uklanjanja te poruke iz reda za čekanje
\item Preuzimanja ulazne datoteke iz deljene memorije i postavljanje u lokalnu
  memoriju.
\item Kreiranje izlazne datoteke u lokalnoj memoriji.
\item Instanciranje novog objekta zadatka tako što su prosleđene putanje do te
  dve datoteke.
\item Pokratanje obrade.
\item Po završetku obrade očekivano je da su rezultat ili poruka o grešci
  upisani u izlaznu datoteku. Ova datoteka je potom postavljena u deljenu
  memoriju.
\item Lokalne datoteke (izlazna i ulazna) su uklonjene.
\item Klijent koji je izdao zahtev za obradu je obavešten o završetku obrade.
  Povratnim kanalom mu je prosleđena putanja do izlazne datoteke u deljenoj memoriji.
\end{enumerate}

{\bf NeboJob} je programska sprega koja ograničava svaki korisnički program. Ova
sprega je opisana putanjom do ulazne i izlazne datoteke i metodom za samu
obradu. Ovakva sprega omogućava korisnicima da sami definišu programe za obradu
proizvoljne složenosti, a ipak ih oslobađa svih režijskih procedura koje su već
ugrađene u samu NeboServer komponentu.

\subsection{nebo.cli}

Komponente iz nebo.cli paketa se direktno oslanjaju i oslikavaju komponente iz
nebo.core paketa. U ovom paketu je opisana tekstualna korisnička sprega
zasnovana na standardnoj Python 3 biblioteci - argparse.

\section{Testiranje i verifikacija}

Ceo projekat je pokriven funkcionalnim (end-to-end) testovima zapisanih u
obliku bash skripti. Napisan je jednostavan pokretač testova koji se zaustavlja
u slučaju greške i ispisuje detaljan izveštaj. Predviđeno je da jedan test bude
opisan u jednoj datoteci. Dostupni su sledeći testovi:

\begin{itemize}
\item {\bf 000 - Smoke test}, provera ispravnosti međuzavisnosti između paketa u
  celom sistemu. Brz test koji proverava najjednostavnije propuste, da li sistem uopšte
  može da se pokrene.
  \item {\bf 001 - Usage}, provera ispravnosti komandi i poruke za pomoć.
  \item {\bf 002 - Start service}, provera ispravnosti komandi za pokretanje i
    zaustavljanje servisa.
  \item {\bf 003 - Ping test}, pokretanje servisa uz kreiranje EC2 instance i
    provera veze do te instance slanjem ping paketa.
  \item {\bf 004 - Upload script}, provera prisutnosti korisničkog programa u
    deljenoj memoriji pri pokretanju servisa.
  \item {\bf 005 - Word count}, potpuna provera ispravnosti servisa na primeru
    programa za brojanje reči. Uključuje proveru rezultata.
  \item {\bf 006 - IO Cleanup}, provera ispravnosti uklanjanja zaostalih
    datoteka u deljenoj memoriji.
\end{itemize}

Pokretanje svih testova je moguće putem komande make test. Moguće je suziti
izbor testova regularnim izrazom zadatim u FILTER argumentu. Na primer make test FILTER=count.

\section{Organizacija repozitorijuma i pakovanje koda}

Repozitorijum je organizovan tako da predstavlja Python paket koji je moguće
instalirati i distribuirati. Instalacija je moguća pomoću alata pip, a moguće ga
je postaviti i u centralni repozitorijum Python paketa (PyPI). Pojedinosti ovog
paketa su određene u datoteci setup.py u korenu repozitorijuma. Pri instalaciji
nebo paketa postavlja se i izvršna datoteka pod nazivom nebo, a koja je
definisana \_\_main\_\_.py datotekom u nebo poddirektorijumu. Sve zavisnosti su
navedene u requirements.txt datoteci, kao standardni način za navođenje zavisnosti.

Sam izvorni kod se nalazi u nebo direktorijumu. Podela je izvršena po paketima,
a svaki direktorijum je i Python paket. To je određeno \_\_init\_\_.py datotekama.
Paketi (moduli) u okviru nebo paketa oslikavaju arhitekturu sistema. To su aws,
cli i core paketi. Pored ova tri osnovna paketa, postoji i data paket koji ne
sadrži Python izvršni kod. Ovde se nalaze samo init\_setup.sh.j2 skripta i primer
posla (test\_job.py). Ove datoteke se distribuiraju uz izvorni kod i važni su za
rad sistema. 

Dokumentacija (ovaj dokument) se nalazi u docs poddirektorijumu. Dokumentaciju
je moguće generisati make docs komandom. Iskorišćen je LaTeX sistem za rad sa
dokumentima i neophodan je LaTeX prevodilac.

Testovi se nalazi unutar tests direktorijuma. Ovaj direktorijum je podeljen na
dva dela, podatke (data) i funkcionalne testove (functional). Polazna tačka za
pokretanje testova je nebo/tests/functional/run.sh skripta a testovi se nalaze
unutar suite direktorijuma. Izlazni kod skripte (exit code) određuje da li je
test uspešan ili nije.

Rad na projektu je olakšan Makefile datotekom u kojoj su definisani često
potrebni zadaci. Omogućeni zadaci su:

\begin{itemize}
  \item test - izvršavanje funkcionalnih testova (uz FILTER parametar)
  \item entr - izvršavanje testova nakon svake promene (uz FILTER parametar,
    pomoću alata entr)
  \item release - objavljivanje nove verzije projekta u namenjenoj unapred
    određenoj S3 korpi. Ovaj korak je važan za rad init\_setup.sh skripte.
  \item docs - generisanje dokumentacija
\end{itemize}

\section{Primer upotrebe}

Prvo je potrebno implementirati novu proceduru za obradu podataka. U nastavku se
nalazi jedan primer.

\mbox{
\lstinputlisting[language=Python,caption=Primer korisničke procedure za obradu podataka]{../tests/data/word_count_script.py}
}

\section{Nedostaci i ideje za dalji razvoj}

Pojedine parametre nije moguće postaviti pomoću tekstualnog korisničkog
interfejsa. Na primer, nije moguće izabrati otisak operativnog sistema, tip
instance, frekvenciju i period zastarenja pri čišćenju zaostalih podataka, tačan
prefiks za naziva S3 korpi i redova za razmenu poruka i drugi. Podrška za
postavljanje svih ovih parametara postoji u kodu, a Python biblioteka argparse
omogućava proširenje korisničkog interfejsa bez većih izmena. Ove izmene samo
nisu bile na najvišem prioritetu i ostavljene su kao mogućnost za dalji razvoj.

Funkcionalni testovi proveravaju ispravnost celog sistema, ali troše resurse.
Iako su ovakvi testovi važni za proveru ispravnosti, uvođenje finijih testova
(unit testovi) bi doprinelo proveri pouzdanosti sistema, smanjilo potrošnju
resursa i ubrzalo razvoj. Jedan od problema koji je potrebno rešiti pre uvođenja
ovakvih testova je maskiranje (eng. mock) Amazon servisa.

Ograničen nivo zaštite od neispravnih korisničkih procedura za obradu postoji.
Ova zaštita se sastoji od provere izuzetaka pri obradi jednog zahteva. Ukoliko
pri obradi jednog zahteva dođe do problema, sledeći zahtev će ipak moći da bude
obrađen. Prostor za unapređenje postoji u smislu zaštite od zlonamernih
procedura za obradu. Na primer, obrada koja se nikad ne završava, obrada koja
zauzima svu ili veliki deo lokalne memorije, obrada koja traje predugo i zauzima
procesorsko vreme ili radnu memoriju. Ovakav vid zaštite i izolacije je moguće
izvesti na primer pomoću mogućnosti Linux jezgra, pre svega kontrolnih grupa
(cgroups) i imenskih prostora (namespaces), po ugledu na popularne Linux
kontejnere.

Jedno važno unapređenje je evidencija o pokrenutim servisima i instancama. Ova
evidencija bi trebala da bude vidljiva svim korisnicima i administratorima
sistema. Jedno predloženo rešenje je da se iskoriste S3 korpe i u njih smeštaju
JSON datoteke. Ovakva evidencij nije bezbedna za konkurentni pristup, ali može
da predstavi prednosti koje donosi ovakva mogućnost. Pouzdaniji način je da se
iskoristi baza podataka koja je predviđena za konkurentni pristup. Uvođenje
evidencije o zauzetim resursima potom otvara vrata i za propisno uklanjanje svih
zauzetih resursa (zauzetih kanala za razmenu poruka i S3 korpi).

\section{Zaključak}

\end{document}